\iffalse

MIT License

Copyright (c) 2023 Aron Hardeman

Permission is hereby granted, free of charge, to any person obtaining a copy
of this software and associated documentation files (the "Software"), to deal
in the Software without restriction, including without limitation the rights
to use, copy, modify, merge, publish, distribute, sublicense, and/or sell
copies of the Software, and to permit persons to whom the Software is
furnished to do so, subject to the following conditions:

The above copyright notice and this permission notice shall be included in all
copies or substantial portions of the Software.

THE SOFTWARE IS PROVIDED "AS IS", WITHOUT WARRANTY OF ANY KIND, EXPRESS OR
IMPLIED, INCLUDING BUT NOT LIMITED TO THE WARRANTIES OF MERCHANTABILITY,
FITNESS FOR A PARTICULAR PURPOSE AND NONINFRINGEMENT. IN NO EVENT SHALL THE
AUTHORS OR COPYRIGHT HOLDERS BE LIABLE FOR ANY CLAIM, DAMAGES OR OTHER
LIABILITY, WHETHER IN AN ACTION OF CONTRACT, TORT OR OTHERWISE, ARISING FROM,
OUT OF OR IN CONNECTION WITH THE SOFTWARE OR THE USE OR OTHER DEALINGS IN THE
SOFTWARE.

\fi
\section{Differential equations (DEs)}
\subsection{First order differential equations}
\tableofcontents[currentsection,currentsubsection]
\begin{frame}{Linear first order differential equations}
    \begin{tcolorbox}[colback=blue!5,colframe=blue!75!black,title=Linear first order DE]
In order to solve the linear differential equation \[y'+P(x)y=Q(x)\]

multiply both sides by $e^{\int P(x)dx}$ (the integrating factor), rewrite using the product rule for derivatives, and integrate both sides.
\end{tcolorbox}
\begin{itemize}
    \item Question: solve $y'+2x y=x$
    \item Solution: next slide.
\end{itemize}
\end{frame}

\begin{frame}{Example: linear 1st order DEs}
    \begin{itemize}
        \item\pause Question: solve $y'+2x y=x$
        \item\pause Solution: the integrating factor is $e^{\int 2xdx}=e^{x^2}$ (the constant of integration in the exponent is omitted). So we multiply both sides of the DE by $e^{x^2}$ and obtain \[e^{x^2}y'+2xe^{x^2}y=xe^{x^2}\]
        Using the product rule for derivatives, this can be rewritten as \[\left[e^{x^2}y\right]'=xe^{x^2}\]
        We integrate both sides and obtain \[e^{x^2}y=\int xe^{x^2} dx = \frac{1}{2}e^{x^2}+C\vspace{-1.5mm}\]
        \vspace{-1.5mm}We can divide both sides by $e^{x^2}$ to find the solution $\boxed{y=\frac{1}{2}+Ce^{-x^2}}$
    \end{itemize}

\end{frame}

\begin{frame}{Separable first order differential equations}
    \begin{itemize}
        \item A \textbf{separable} first order differential equation is an equation where the x's and y's can be ``separated", thus the equation can be rewritten into the form $y'f(y)=g(x)$. The method to solve these is to rewrite that equation to the form $f(y)dy=g(x)dx$ and then integrate both sides $\int f(y)dy=\int g(x)dx$:
        \item\pause Question: $xyy'=x^2+1$
        \item\pause Solution: rewrite the equation into $ydy=\frac{x^2+1}{x}dx$ and integrate both sides: $\int ydy=\int\frac{x^2+1}{x}dx$\pause

        So we find $\frac{1}{2}y^2=\frac{1}{2}x^2+\ln\left|x\right|+C$ \pause
        
        So the final answer becomes $\boxed{y=\pm\sqrt{x^2+2\ln \left|x\right| + C_*}}$ where $C_*=2C$.
    \end{itemize}

\end{frame}


\subsection{Second order differential equations}
\tableofcontents[currentsection,currentsubsection]

\begin{frame}

\frametitle{(Linear) second order differential equations}


\begin{tcolorbox}[colback=blue!5,colframe=blue!75!black,title=General form]
Homogeneous: $ay''+by'+cy=0$

Non-homogeneous: $ay''+by'+cy=f(x)$

{\footnotesize (here we will only deal with the case that $a,b$ and $c$ are constant real numbers)}
\end{tcolorbox}

\pause
Example:
\[y''-8y'+15y=0\]

\end{frame}
\begin{frame}
\frametitle{Solving a homogeneous 2nd order DE: basic example}
Example:
$y''-8y'+15y=0$

\pause

First construct the corresponding quadratic equation and solve it:
\begin{flalign*}
&r^2-8r+15=0\\ 
&(r-3)(r-5)=0\\
&r=3 \lor r=5\\
\end{flalign*}

\pause
We have two distinct real roots (3 and 5), so the general solution of the DE is
\[\boxed{y=c_1e^{3x}+c_2e^{5x}}\hspace{10mm}\text{for any constants $c_1$ and $c_2$}\]
\end{frame}




\begin{frame}
\frametitle{``Algorithm" to solve the homogeneous case}
You have some DE which you want to solve:
$\pmb{ay''+by'+cy=0}$

\begin{itemize}
\item Step 1: construct the characteristic equation: $ar^2+br+c=0$\pause
\item Step 2: solve it (compute the roots/solutions $r_1$ and $r_2$)\pause
\item Step 3: use the following scheme to find your final answer:
\begin{exampleblock}{Solution cases for homogeneous 2nd order DE}
\begin{itemize}
\item Real (non-equal) roots: \hspace{8.7mm} $y_{(x)}=c_1e^{r_1 x}+c_2e^{r_2x}$
\item One real root: \hspace{22mm} $y_{(x)}=c_1e^{rx}+c_2xe^{rx}$
\item Case $r_{1,2}=\alpha\pm i\beta$: \hspace{14mm} $y_{(x)}=e^{\alpha x}[c_1\cos (\beta x)+ c_2 \sin (\beta x)]$

\end{itemize}
\end{exampleblock}

\end{itemize}

\end{frame}






\begin{frame}
\frametitle{Solving a homogeneous 2nd order DE: another example}
{\small Example:
$7y''-7y'+2y=0$

\pause

First construct the corresponding quadratic equation and solve it:
\begin{flalign*}
&7r^2-7r+2=0\\ 
&r_{1,2}=\frac{-(-7)\pm\sqrt{(-7)^2-4\cdot7\cdot2}}{2\cdot7}\\
&r_{1,2}=\frac{7\pm\sqrt{-7}}{14}\\
&r_{1,2}=\frac{1}{2}\pm i\frac{\sqrt{7}}{14}\\
\end{flalign*}

\pause
(Recall: Case $r_{1,2}=\alpha\pm i\beta$: \quad $y_{(x)}=e^{\alpha x}[c_1\cos (\beta x)+ c_2 \sin (\beta x)]$)

\pause
We have two complex roots, so the general solution of the DE is
\[{y=e^{\frac{1}{2} x}\left[c_1\cos\left(\frac{\sqrt{7}}{14} x\right) + c_2\sin\left(\frac{\sqrt{7}}{14} x\right)\right]}\hspace{7mm}\text{for any constants $c_1$ and $c_2$}\]
}
\end{frame}



\begin{frame}
\frametitle{Solving a homogeneous 2nd order DE: one more example}
Example:
$y''+16y'+64y=0$

\pause

First construct the corresponding quadratic equation and solve it:
\begin{flalign*}
&r^2+16r+64=0\\
&(r+8)^2=0\\
&r=-8\\
\end{flalign*}

\pause
We have just one root this time!

(Recall: in case there's just one root: $y_{(x)}=c_1e^{r x}+c_2xe^{rx}$)

\pause
Therefore the general solution is:
\[\boxed{y=c_1e^{-8x}+c_2\pmb{x}e^{-8x}}\hspace{10mm}\text{for any constants $c_1$ and $c_2$}\]
Beware: in case there is only one root, multiply the second term (xor the first term) with $x$!
\end{frame}


%\subsubsection{Non-homogeneous}
\begin{frame}
\frametitle{NON-homogeneous second order DEs}

\begin{tcolorbox}[colback=blue!5,colframe=blue!75!black,title=General form]
$\pmb{ay''+by'+cy=f(x)}$

{\small ($a,b$ and $c$ are constant real numbers)}
\end{tcolorbox}

Plan of attack:
\begin{itemize}
\pause\item Step 1: consider the complementary equation $ay''+by'+cy=\pmb{0}$ and compute it's solution $y_c$. {\small(This is easy as it's a homogeneous equation)}
\pause\item Step 2: find some particular solution $y_p$ to the original non-homogeneous equation
\pause\item Step 3: your general solution to the original equation is now $y=y_c+y_p$
\end{itemize}
The difficulty may be mostly in step 2.
\end{frame}

\begin{frame}
\frametitle{Non-homogeneous second order DEs: example 1 (part 1)}


\begin{itemize}
\item Question: find the general solution of the differential equation $7y''-7y'+2y=x^2+7$.
\pause\item Step 1: we had already found the complementary solution (to the equation $7y''-7y'+2y=0$) before: ${y_c=e^{\frac{1}{2} x}\left[c_1\cos\left(\frac{\sqrt{7}}{14} x\right) + c_2\sin\left(\frac{\sqrt{7}}{14} x\right)\right]}$ for any constants $c_{1,2}$.
\pause\item Step 2: we must find some particular solution. Since $x^2+7$ is a 2nd order polynomial, let's set our particular solution to $y_p=Ax^2+Bx+C$. We plug this in the DE in order to find $A,B$, and $C$. 
\pause So we have $y_p=Ax^2+Bx+C$, $y_p'=2Ax+B$ and $y_p''=2A$. Let's plug this in:
\end{itemize}
\end{frame}

\begin{frame}
\frametitle{Non-homogeneous second order DEs: example 1 (part 2)}


\begin{itemize}
\item We will plug $y_p=Ax^2+Bx+C$, $y_p'=2Ax+B$ and $y_p''=2A$ in the original DE ($7y''-7y'+2y=x^2+7$) to find $A$, $B$ and $C$ of the particular solution:
\[7(2A)-7(2Ax+B)+2(Ax^2+Bx+C)=x^2+7\]
\[(2A)x^2+(-14A+2B)x+(14A-7B+2C)=x^2+7\]

\pause This must hold for all $x$, so the coefficients of the polynomials on the left- and right-hand side, must be equal. So, we have $2A=1$, and $-14A+2B=0$, and $14A-7B+2C=7$. \pause From the first one, we find $A=\frac{1}{2}$, then from the second one we find $B=\frac{7}{2}$, after which the third one gives us $C=\frac{49}{4}$. \pause Thus, we've found a particular solution: $y_p=\frac{1}{2}x^2+\frac{7}{2}x+\frac{49}{4}$.
\end{itemize}
\end{frame}

\begin{frame}
\frametitle{Non-homogeneous second order DEs: example 1 (part 3)}


\begin{itemize}
\item Step 3: now that we have found the complementary solution ${y_c=e^{\frac{1}{2} x}\left[c_1\cos\left(\frac{\sqrt{7}}{14} x\right) + c_2\sin\left(\frac{\sqrt{7}}{14} x\right)\right]}$ and a particular solution $y_p=\frac{1}{2}x^2+\frac{7}{2}x+\frac{49}{4}$, we can simply add them up to obtain the general solution of $7y''-7y'+2y=x^2+7$:

\[\boxed{{y=e^{\frac{1}{2} x}\left[c_1\cos\left(\frac{\sqrt{7}}{14} x\right) + c_2\sin\left(\frac{\sqrt{7}}{14} x\right)\right]}+\frac{1}{2}x^2+\frac{7}{2}x+\frac{49}{4}}\]
for any constants $c_1$ and $c_2$.
\end{itemize}
\end{frame}

\begin{frame}
\frametitle{The method of undetermined coefficients (formal)}{\scriptsize
\begin{itemize}
    \item In the previous example, we had $x^2+7$ (a polynomial of order 2) on the right-hand side of the differential equation. So we guessed that a particular solution could be a polynomial of order 2 as well ($Ax^2+Bx+C$). In general:\pause

    \begin{tcolorbox}[colback=blue!5,colframe=blue!75!black,title=Method of undetermined coefficients (FORMAL)]
    We search a particular solution to the differential equation $ay''+by'+cy=f(x)$.

    Let $P_n(x)$ and $Q_n(x)$ and $R_n(x)$ denote polynomials of order $n$.
    \begin{itemize}
        \item If $f(x)=e^{kx}P_n(x)$, then try $y_p=e^{kx}Q_n(x)$.
        \item If $f(x)=e^{kx}P_n(x)\sin mx$ or $f(x)=e^{kx}P_n(x)\cos mx$, then try $y_p=e^{kx}Q_n(x)\cos mx + e^{kx}R_n(x)\sin mx$
    \end{itemize}
    If any term in your ``guess" is a solution to the complementary equation, then multiply your guess $y_p$ by $x$ (or $x^2$ if it's still the case). 
    
    Plug your $y_p$-guess in the DE in order to find the coefficients of $Q_n(x)$ and $R_n(x)$.
    
    
    \end{tcolorbox}
    In the previous example we had the first case (with $k=0$ such that $e^{kx}=1$).

    
\end{itemize}}
\end{frame}

\begin{frame}
\frametitle{The method of undetermined coefficients (examples)}{\footnotesize
\begin{itemize}
    \item We search a particular solution to $ay''+by'+cy=f(x)$.
    \item If $f(x)=x^3$ or $f(x)=10000x^3+x+12$, we would try $y_p=Ax^3+Bx^2+Cx+D$.
    \item If $f(x)=\sin 8x$ or $f(x)=137\cos 8x$, we would try $y_p=A\cos 8x + B\sin 8x$.
    \item If $f(x)=e^{7x}$ or $f(x)=39e^{7x}$, we would try $y_p=Ae^{7x}$.
    \item If $f(x)=xe^{8x}$ or $f(x)=xe^{8x}+e^{8x}$, we would try $y_p=(Ax+B)e^{8x}$.
    \item If $f(x)=x^2\sin {x}$, we would try $y_p=(Ax^2+Bx+C)\cos x + (Dx^2+Ex+F)\sin x$.
    \item If $f(x)=e^{9x}x^2\sin 4x$, we would try $y_p=e^{9x}(Ax^2+Bx+C)\cos 4x + e^{9x}(Dx^2+Ex+F)\sin 4x$.
    \item Notice that the last two examples are so long that you will probably not get them on your exam (since you'd have to solve for 6 coefficients). However they are useful as a demonstration of the principle.
    \item \textbf{Do not forget that you have to multiply your $y_p$-guess by $x$ if any term in your guess is a solution to the complementary equation.}
    
\end{itemize}}
\end{frame}

\begin{frame}{The superposition principle}
\[ay''+by'+cy=f_1(x)+f_2(x)\]
\begin{itemize}
    \item Sometimes, $f(x)$ is a sum of multiple functions, say $f(x)=f_1(x)+f_2(x)$. In that case, you can just find a particular solution $y_{p1}$ to the differential equation $ay''+by'+cy=f_1(x)$ and a particular solution $y_{p2}$ to the differential equation $ay''+by'+cy=f_2(x)$.

    \item Your particular solution to the differential equation $ay''+by'+cy=f_1(x)+f_2(x)$ is then given by $y_{p1}+y_{p2}$.
    \item Do not forget to add the complementary solution to your answer as well.

    \item (This also works for a sum of more than two functions; see next slide for a full example.)
    \end{itemize}
\end{frame}

\begin{frame}{Superposition principle \& method of u.c. (example)}

{\scriptsize
\begin{itemize}
    \item Question: solve $y''-6y'+8y=xe^{3x}+xe^{4x}+xe^{5x}$.
    \item The complementary solution is $\textcolor{teal}{y_c=c_1e^{2x}+c_2e^{4x}}$ for any constants $c_1$ and $c_2$.
    \item Let $\textcolor{red}{y_{p1}}$ be a particular solution to $y''-6y'+8y=xe^{3x}$. Then $\textcolor{red}{y_{p1}}$ must be of the form $\textcolor{red}{y_{p1}=(Ax+B)e^{3x}}$. Substituting this in $y''-6y'+8y=xe^{3x}$ gives that $A=-1$ and $B=0$. So we find $\textcolor{red}{y_{p1}=-xe^{3x}}$.
    \item Let $\textcolor{blue}{y_{p2}}$ be a particular solution to $y''-6y'+8y=xe^{4x}$. Then $\textcolor{blue}{y_{p2}}$ would be of the form $\textcolor{gray}{y_{p2}=(Cx+D)e^{4x}}$, but we observe that the term $De^{4x}$ is a solution to the complementary equation (since $y_c=c_1e^{2x}+c_2e^{4x}$), thus we multiply the $\textcolor{blue}{y_{p2}}$-guess by $x$ and obtain $\textcolor{blue}{y_{p2}=(Cx^2+Dx)e^{4x}}$. We substitute this into $y''-6y'+8y=xe^{4x}$ and obtain $C=\frac{1}{4}$ and $D=-\frac{1}{4}$, so we find $\textcolor{blue}{y_{p2}=(\frac{1}{4}x^2-\frac{1}{4}x)e^{4x}}$.
    \item Let $\textcolor{violet}{y_{p3}}$ be a particular solution to $y''-6y'+8y=xe^{5x}$. Then $\textcolor{violet}{y_{p3}}$ must be of the form $\textcolor{violet}{y_{p3}=(Ex+F)e^{5x}}$. Substituting this in $y''-6y'+8y=xe^{5x}$ gives that $E=\frac{1}{3}$ and $F=-\frac{4}{9}$. So we find $\textcolor{violet}{y_{p3}=(\frac{1}{3}x-\frac{4}{9})e^{5x}}$.
    \item The particular solution to the original differential equation is now $\textcolor{red}{y_{p1}}+\textcolor{blue}{y_{p2}}+\textcolor{violet}{y_{p3}}=\textcolor{red}{-xe^{3x}}+\textcolor{blue}{(\frac{1}{4}x^2-\frac{1}{4}x)e^{4x}}+\textcolor{violet}{(\frac{1}{3}x-\frac{4}{9})e^{5x}}$. We add the full particular solution to the complementary solution and obtain as our final answer:
    \[\boxed{y=\textcolor{teal}{c_1e^{2x}+c_2e^{4x}}\textcolor{red}{-xe^{3x}}\textcolor{blue}{+\left(\frac{1}{4}x^2-\frac{1}{4}x\right)e^{4x}}\textcolor{violet}{+\left(\frac{1}{3}x-\frac{4}{9}\right)e^{5x}}}\qquad\text{for all $c_1$ and $c_2$}\]
    \end{itemize}
}\end{frame}

\begin{frame}
\frametitle{Sample question on differential equations (slide 1)}
\textbf{Question:} solve the initial value problem \[y''+2y'-35y=3e^{5x}\qquad y(0)=137\qquad y'(0)=42\]

\textbf{Solution steps:}
\begin{itemize}
    \item\pause Step 1: solve the homogeneous equation $y''+2y'-35y=0$ to find the complementary solution.
    \item\pause Step 2: use the method of undetermined coefficients to find a particular solution to the original (non-homogeneous) equation.
    \item\pause Step 3: we add the complementary solution to the particular solution to find the general solution of the original equation.
    \item\pause Step 4: apply the initial values to obtain the final answer.
    \item\pause (Fully worked out solution on next slides)
\end{itemize}
\end{frame}

\begin{frame}{Sample question on differential equations (slide 2)}
    \textbf{Question:} solve the initial value problem \[y''+2y'-35y=3e^{5x}\qquad y(0)=137\qquad y'(0)=42\]
    
    \textbf{Step 1:} first we solve $y''+2y'-35y=0$. \pause The characteristic equation is $r^2+2r-35=0$, thus $(r+7)(r-5)=0$, so the roots are $5$ and $-7$, two distinct real numbers.\pause

    Thus, the complementary solution takes the form $y_c = c_1e^{5x} + c_2e^{-7x}$ for any constants $c_1$ and $c_2$. (Later we will determine which $c_1$ and $c_2$ suit our initial values.)
\end{frame}

\begin{frame}{Sample question on differential equations (slide 3)}
    \textbf{Question:} solve the initial value problem \[y''+2y'-35y=3e^{5x}\qquad y(0)=137\qquad y'(0)=42\]
    
    \textbf{Step 2:} we apply the method of undetermined coefficients as explained before. $f(x)=3e^{5x}$, so we would try the particular solution $\textcolor{gray}{y_p=Ae^{5x}}$.
    \begin{itemize}
    \item{\scriptsize(Recall the first case from the method of u.c.: if $f(x)=e^{kx}P_n(x)$, then try $y_p=e^{kx}Q_n(x)$. Here $P_n(x)=3$, a ``polynomial" of degree 0)}
    \end{itemize}\pause
    However, the complementary solution was $y_c = c_1e^{5x} + c_2e^{-7x}$ for any constants $c_1$ and $c_2$. We observe that our trial particular solution $\textcolor{gray}{y_p=Ae^{5x}}$ will not work, because it is a solution to the complementary equation! Thus, we multiply our guess by $x$, so our trial particular solution is $\pmb{y_p=Axe^{5x}}$, but we still need to find the constant $A$ (next slide).

\end{frame}

\begin{frame}{Sample question on differential equations (slide 4)}
    \textbf{Question:} solve the initial value problem \[y''+2y'-35y=3e^{5x}\qquad y(0)=137\qquad y'(0)=42\]
    
    \textbf{Step 2 (continuation):} our trial particular solution is $y_p=Axe^{5x}$, but we need to find $A$. So we compute the derivatives: $y_p'=A(e^{5x}+5xe^{5x})$ and $y_p''=A(5e^{5x}+5e^{5x}+25xe^{5x})=A(10e^{5x}+25xe^{5x})$\pause

    We substitute this in the original differential equation to find: \[A(10e^{5x}+25xe^{5x})+2A(e^{5x}+5xe^{5x})-35Axe^{5x}=3e^{5x}\] \vspace{-7mm}\[\iff 12Ae^{5x}=3e^{5x}\]\pause
    So we take $A=\frac{1}{4}$. The guess worked (since we were able to find an $A$ such that $y_p=Axe^{5x}$ satisfies the differential equation), so we found the valid particular solution $\pmb{y_p=\frac{1}{4}xe^{5x}}$.

    

\end{frame}




\begin{frame}{Sample question on differential equations (slide 5)}
    \textbf{Question:} solve the initial value problem \[y''+2y'-35y=3e^{5x}\qquad y(0)=137\qquad y'(0)=42\]
    
    \textbf{Step 3):} the general solution to the complementary equation was $y_c = c_1e^{5x} + c_2e^{-7x}$ and a particular solution is $y_p=\frac{1}{4}xe^{5x}$. We add these together to obtain the general solution to the non-homogeneous (original) equation for any constants $c_1$ and $c_2$:\pause

    \[y=c_1e^{5x} + c_2e^{-7x} +\frac{1}{4}xe^{5x}\]\pause

    This is a solution for every $c_1$ and $c_2$, but we were given an initial value problem, i.e. we still have to find $c_1$ and $c_2$ such that $y(0)=137$ and $y'(0)=42$ (step 4, next slide).
    

\end{frame}


\begin{frame}{Sample question on differential equations (slide 6)}
    \textbf{Question:} solve the initial value problem \[y''+2y'-35y=3e^{5x}\qquad y(0)=137\qquad y'(0)=42\]
    
    \textbf{Step 4):} the general solution to the differential equation is
    $y=c_1e^{5x} + c_2e^{-7x} +\frac{1}{4}xe^{5x}$\pause, with derivative $y'=5c_1e^{5x}-7c_2e^{-7x}+\frac{1}{4}e^{5x}+\frac{5}{4}xe^{5x}$.

    We need to have $y(0)=137$ and $y'(0)=42$\pause, i.e.
    \[y(0)=c_1+c_2=137\qquad y'(0)=5c_1-7c_2+\frac{1}{4}=42\]\pause
    Substituting $c_2=137-c_1$ into the second equation gives $5c_1-7(137-c_1)+\frac{1}{4}=42 \iff 12c_1=\frac{4003}{4} \iff c_1=\frac{4003}{48}$ and from we first equation we obtain $c_2=137-\frac{4003}{48}=\frac{2573}{48}$. Therefore, the solution is \[\boxed{\boxed{y=\frac{4003}{48}e^{5x} + \frac{2573}{48}e^{-7x} +\frac{1}{4}xe^{5x}}}\]

    

\end{frame}









































\iffalse

\begin{frame}{Sample question on differential equations (slide 2)}
    \textbf{Question:} solve the initial value problem \[y''+2y'-35y=3e^{3x}\qquad y(0)=137\qquad y'(0)=42\]
    
    \textbf{Step 1:} first we solve $y''+2y'-35y=0$. The characteristic equation is $r^2+2r-35=0$, thus $(r+7)(r-5)=0$, so the roots are $-5$ and $7$, two distinct real numbers.

    Thus, the complementary solution takes the form $y_c = c_1e^{-5x} + c_2e^{7x}$ for any constants $c_1$ and $c_2$. (Later we will determine which $c_1$ and $c_2$ suit our initial values.)
\end{frame}

\begin{frame}{Sample question on differential equations (slide 3)}
    \textbf{Question:} solve the initial value problem \[y''+2y'-35y=3e^{3x}\qquad y(0)=137\qquad y'(0)=42\]
    
    \textbf{Step 2:} we apply the method of undetermined coefficients as explained before. $f(x)=3e^{3x}$, so we would try the particular solution $y_p=Ae^{3x}$.
    \begin{itemize}
    \item{\scriptsize(Recall the first case from the method of u.c.: if $f(x)=e^{kx}P_n(x)$, then try $y_p=e^{kx}Q_n(x)$. Here $P_n(x)=3$, a ``polynomial" of degree 0)}
    \end{itemize}
    The derivatives are $y_p'=3Ae^{3x}$ and $y_p''=9Ae^{3x}$, and we plug this in the differential equation to obtain
    \[9Ae^{3x}+2(3Ae^{3x})-35Ae^{3x}=3e^{3x}\iff-20Ae^{3x}=3e^{3x}\]
    So we take $A=-\frac{3}{20}$, (our guess worked,) thus we found a valid particular solution $\pmb{y_p=-\frac{3}{20}e^{3x}}$.

\end{frame}

\begin{frame}{Sample question on differential equations (slide 4)}
    \textbf{Question:} solve the initial value problem \[y''+2y'-35y=3e^{3x}\qquad y(0)=137\qquad y'(0)=42\]
    
    \textbf{Step 3:} we add the complementary solution to the particular solution and obtain as the general solution to the differential equation: \[y=c_1e^{-5x} + c_2e^{7x}-\frac{3}{20}e^{3x}\] for any constants $c_1$ and $c_2$. However, the question was an initial value problem; we still have to determine $c_1$ and $c_2$ such that the initial values hold.

\end{frame}

\begin{frame}{Sample question on differential equations (slide 5)}
    \textbf{Question:} solve the initial value problem \[y''+2y'-35y=3e^{3x}\qquad y(0)=137\qquad y'(0)=42\]
    
    \textbf{Step 4:} we have the general solution and compute its derivative: \[y(x)=c_1e^{-5x} + c_2e^{7x}-\frac{3}{20}e^{3x}\]  \vspace{-5mm}\[y'(x)=-5c_1e^{-5x}+7c_2e^{7x}-\frac{9}{20}e^{3x}\]
    We apply the initial conditions: \[y(0)=c_1+c_2-\frac{3}{20}=137\quad\quad y'(0)=-5c_1+7c_2-\frac{9}{20}=42\]
    

\end{frame}
\fi