\iffalse

MIT License

Copyright (c) 2023 Aron Hardeman

Permission is hereby granted, free of charge, to any person obtaining a copy
of this software and associated documentation files (the "Software"), to deal
in the Software without restriction, including without limitation the rights
to use, copy, modify, merge, publish, distribute, sublicense, and/or sell
copies of the Software, and to permit persons to whom the Software is
furnished to do so, subject to the following conditions:

The above copyright notice and this permission notice shall be included in all
copies or substantial portions of the Software.

THE SOFTWARE IS PROVIDED "AS IS", WITHOUT WARRANTY OF ANY KIND, EXPRESS OR
IMPLIED, INCLUDING BUT NOT LIMITED TO THE WARRANTIES OF MERCHANTABILITY,
FITNESS FOR A PARTICULAR PURPOSE AND NONINFRINGEMENT. IN NO EVENT SHALL THE
AUTHORS OR COPYRIGHT HOLDERS BE LIABLE FOR ANY CLAIM, DAMAGES OR OTHER
LIABILITY, WHETHER IN AN ACTION OF CONTRACT, TORT OR OTHERWISE, ARISING FROM,
OUT OF OR IN CONNECTION WITH THE SOFTWARE OR THE USE OR OTHER DEALINGS IN THE
SOFTWARE.

\fi
\section{Complex numbers}
\tableofcontents[currentsection,currentsubsection]
\begin{frame}
\frametitle{Basic arithmetic}

Complex numbers $(a+bi$ with $a,b\in\mathbb{R})$ behave just like you'd expect when doing simple arithmetic. For example:

\pause $(12-23i) + (3+6i) = 15 -17i$

\pause $(12-23i) - (3+6i) = 9 -29i$

\pause $(5+6i)(7+8i) = 35+82i+48i^2 = -13+82i$

\pause For division, use this trick with the complex conjugate of the denominator:

$\frac{2-3i}{4-5i}=$\pause$\frac{2-3i}{4-5i}\cdot\underbrace{\frac{4+5i}{4+5i}}_{=1} $\pause$= \frac{(2-3i)(4+5i)}{(4-5i)(4+5i)}$\pause$=\frac{23-2i}{41}$\pause$=\boxed{\frac{23}{41}-\frac{2}{41}i}$


\end{frame}

\begin{frame}
\frametitle{Complex plane}

Complex numbers lie in the \textit{complex plane}.

%\pause\includegraphics[scale=0.8]{plane.png}

\begin{tikzpicture}
\begin{axis}[xlabel=$\mathfrak{Re}$,ylabel=$\mathfrak{Im}$,xmin=-7,ymin=-7,xmax=7,ymax=7,axis lines = middle,width=7cm,height=7cm,yticklabels={,,},xticklabels={,,}]
\addplot[data cs=polar,red,domain=0:360,samples=360,smooth] (x,5);
%\node[label={30:{(3,4)}},circle,fill,inner sep=2pt] at (axis cs:3,4) {};
\node[label={270:{a}},circle,fill,inner sep=1pt] at (axis cs:3,0) {};
\node[label={180:{b}},circle,fill,inner sep=1pt] at (axis cs:0,4) {};
\draw (axis cs:0,0) -- node[label={[xshift=0.1cm, yshift=-0.1cm]$r$}, left]{} (axis cs:3,4);
\draw[dashed] (axis cs:0,4) -- (axis cs:3,4);
\draw[dashed] (axis cs:3,0) -- (axis cs:3,4);
\draw
  (axis cs:5,0) coordinate (a)
  -- (axis cs:0,0) coordinate (b)
  -- (axis cs:3,4) coordinate (c) node[above right] {$z$}
  pic["$\theta$",draw=red,->,angle eccentricity=1.5,angle radius=0.6cm] {angle=a--b--c};
\end{axis}
\end{tikzpicture}
${\boxed{\boxed{z=a+bi=re^{i\theta}=r[\cos\theta+i\sin\theta]}}}$



\pause Instead of writing $a+bi$ (rectangular form), we can equivalently write $re^{i\theta}$ or $r[\cos\theta+i\sin\theta]$. The latter is known as the \textit{polar form}.
\end{frame}

\begin{frame}
\frametitle{Converting to exponential and polar form}
\begin{itemize}
\item The modulus of a complex number $z=x+yi$ is $\boxed{r=|z|=\sqrt{x^2+y^2}}$.

\item\pause In order to find the (principal) argument $\Arg(z)$, use the formula:
\[{\theta=\Arg(z)=\atantwo(y,x)=\begin{cases}
\arctan(\frac{y}{x}) &x>0\\
\arctan(\frac{y}{x})+\pi &x<0, y\geq0\\
\arctan(\frac{y}{x})-\pi &x<0, y<0 \\
\frac{\pi}{2}&x=0, y>0\\
-\frac{\pi}{2} &x=0, y<0\\
\text{(undefined)} &x=0, y=0
\end{cases}}\]

This formula gives the angle between $-\pi<\theta\leq\pi$.

\pause\textbf{Angles are in radians!} Fun fact: the function $\atantwo(y,x)$ is implemented in many programming languages, including C, Java, ...

\end{itemize}
\end{frame}

\begin{frame}{Converting to exponential form (examples)}
    \begin{itemize}
        \item\pause  $1+i=\sqrt{2}\cdot e^{i\pi/4}$ \textcolor{gray}{(since $\sqrt{1^2+1^2}=\sqrt{2}$ and $\Arg(1+i)=\frac{\pi}{4}$)}
        \item\pause  $1-i=\sqrt{2}\cdot e^{-i\pi/4}$ \textcolor{gray}{(now the angle is $\Arg(1-i)=-\frac{\pi}{4}$)}
        \item\pause  $-10=10e^{i\pi}$ \textcolor{gray}{($-10$ lies on the negative real axis, so the angle is $\pi$)}
    \end{itemize}

    
\end{frame}

\begin{frame}
\frametitle{Multiplying two complex numbers (polar form)}
\begin{itemize}
\item Suppose that we have two complex numbers $z_1=r_1 e^{i\theta_1}$ and $z_2=r_2 e^{i\theta_2}$. Let's see what happens when we multiply:
\pause\item $z_1z_2=r_1e^{i\theta_1}r_2 e^{i\theta_2}=r_1r_2e^{i(\theta_1+\theta_2)}$.
\pause\item Thus, when multiplying two complex numbers, the modulus gets multiplied and the arguments (angles) get added.
\end{itemize}
\end{frame}

\begin{frame}
\frametitle{Getting the n$^\text{th}$ roots of a number}
\begin{itemize}
\item Question: find all $z\in\mathbb{C}$ for which $z^5=-10$.
\pause\item Solution: we have that $arg(-10)=\textcolor{blue}{\pi+2k\pi}$ and $|-10|=\textcolor{red}{10}$. \pause So we can write $z^5=-10=\textcolor{red}{10}e^{i(\textcolor{blue}{\pi+2k\pi})}$. {\scriptsize(for $k\in\mathbb{Z}$)}
\pause\item Raising left-hand and right-hand side to the power $\frac{1}{5}$, we obtain $z=10^{1/5}e^{i(\frac{\pi}{5}+\frac{2k\pi}{5})}$
\pause\item We have five solutions, so for example, take $k$ to be $0,1,2,3,4$ to find the following solutions:
\begin{align*}
&z=10^{1/5}e^{i(\frac{\pi}{5})} \quad&\lor\quad   z=10^{1/5}e^{i(\frac{3\pi}{5})} &&\\
\lor\quad & z=10^{1/5}e^{i\pi}=-10^{1/5}  &\lor\quad
z=10^{1/5}e^{i(\frac{7\pi}{5})}   &\quad\lor\quad 
z=10^{1/5}e^{i(\frac{9\pi}{5})} & 
\end{align*}
\pause\item (These are all solutions: if we were to go on for $k=5,6,\hdots$, then the solutions would repeat because sine and cosine (and thus, $e^{i\theta})$ have a period of $2\pi$.)
\end{itemize}
\end{frame}

\begin{frame}
\frametitle{De Moivre's formula}
\begin{itemize}

\item Question: write $(\sqrt{3}+i)^{1000}$ in the form $a+bi$
\pause\item One approach would be to expand brackets a thousand times. However, there is a faster method.
\pause\item We can write $(\sqrt{3}+i)^{1000}=(2e^{i\pi/6})^{1000}=2^{1000}e^{1000i\pi/6}=2^{1000}e^{4i\pi/6}=2^{1000}\left(-\frac{1}{2}+\frac{\sqrt{3}}{2}i\right)=\boxed{-2^{999} +2^{999}\sqrt{3}i} $
\end{itemize}

\begin{tcolorbox}[colback=yellow!50,colframe=violet!85!black,title=De Moivre's formula]
    Suppose we have a complex number $z=e^{i\theta}$. Then, as in the example, $z^n=(e^{i\theta})^n=e^{in\theta}$. Rewriting in polar form gives us \textit{De Moivre's formula}: \[(\cos\theta+i\sin\theta)^n=\cos n\theta + i\sin n\theta\] In practice, using the exponential form (as in the example) may be easier.
    
\end{tcolorbox}
\end{frame}